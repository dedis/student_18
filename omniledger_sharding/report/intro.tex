\section{Introduction} \label{intro}
The recent success and surge of popularity enjoyed by Bitcoin has sparked the curiosity of many. Beyond the economical aspects of the cryptocurrency, lies an intriguing distributed ledger technology capable of processing transactions in a decentralised manner without compromising on security. Traditionally, transaction processing systems were designed with a centralised philosophy in mind: a transaction between two parties is processed by a third party. One of the main problem of this design is the fact that this third party constitutes a single point of failure in the system, which makes it an attractive target for security attacks. In Bitcoin, there is no such central entity. Instead, a large number of nodes participating in the network, called miners, validate transactions concurrently and come to an agreement via the use of Nakamoto's consensus. For these reasons, among others, some consider Bitcoin to offer the first successful decentralised transaction processing system.
Consequently, blockchain technology has gained a lot of interest in recent years. There has been an increasing amount of research on finding the best possible applications for this innovative technology but also new distributed ledger designs aiming to provide interesting features or attempt to correct existing problems. \\\\
One of the key challenges blockchain technology faces today is to offer performance on par with more traditional centralised payment processors (e.g. Visa). In particular, how to achieve scalability, i.e. increase the total transaction volume with the number of participants. Many approaches have been considered, however most of them trade either security or decentralisation for scalability. The next section briefly introduces OmniLedger\cite{kokoris-kogias_omniledger:_2018}, a novel scale-out distributed ledger that preserves long-term security under permissionless operation.

\subsection{OmniLedger}
OmniLedger is a permissionless distributed ledger capable of scaling-out while preserving security. It is designed to achieve higher throughput and shorter confirmation time via sharding. The idea of sharding is to divide the ledger state into multiple parts (shards) and assign to each a different subset of validators. For example, in an account-based ledger, all transactions involving accounts from Europe would be handled by shard \textit{A}, while transactions involving accounts from Asia would be handled by shard \textit{B}. This partitioning will allow each shard to process transactions in parallel. As a result, the individual transaction processing load will be reduced as a validator will only process transactions given to its shard, instead of every transaction. Moreover, the system processing power will increase proportionally with the number of validators as the addition of new participating nodes will results in more shards. \\\\
The main goal is to achieve sharding in a decentralised and secure manner while providing good performance. Here are described briefly a few of the most important challenges in achieving that goal and how OmniLedger overcomes them. First, the sharding assignment must be bias-resistant, the design employs distributed randomness generation protocols (e.g. RandHound\cite{cryptoeprint:2016:1067}) which generate verifiable randomness in a unbiased manner. Such randomness can then be used to generate shard assignments. Next, there needs to be a way to process transactions involving different shards, in particular how to make sure such transactions are either committed or aborted atomically. To that end, OmniLedger introduces Atomix, a two-step cross-shard transactions commit protocol where the idea is to first send the transaction to every input shard, then collect their individual answers, either an acknowledgement or an error. Then, abort if there is an error, otherwise commit the transaction to the output shards. The last challenge is to offer better support for low-value payments. Currently, transactions in blockchain technology systems have high-latency regardless of their value due to eventual consistency (e.g. Bitcoin recommends waiting for six subsequent blocks before considering a transaction as permanently on the blockchain). This is a problem especially for low-value transactions where the stakes are not high enough to warrant such a long wait. The solution OmniLedger proposes is to use a two-tier validation scheme called Trust-but-Verify. In this scheme, transactions are sent from the client to a first tier of validators called optimistic validators, then to a second tier called core validators before finally being added to a block. The difference between the two tiers lies size: optimistic validators form smaller shards than core validators, this allows them to process transactions faster at the expense of some security. Thus, low-payment transactions can already be accepted after the first validation, while seller can choose to wait for both validation steps for higher-stake transactions. \\\\
An implementation of OmniLedger is currently under development in the Decentralized and Distributed Systems lab (DEDIS) at EPFL\cite{dedis}. In its current state, the project supports many features including smart contracts, creation of individual ledgers, multiple transactions per block. However it does not include sharding nor integrate multiple ledgers. \\\\
The goal of this project is to incorporate a first sharding algorithm to the current OmniLedger project. Implementing sharding and all the necessary mechanisms to make it secure and decentralised (as described previously) takes more time than a semester project allows. Thus, this project focuses more on creating a solid foundation, such that future works can continue to build upon it, towards a complete implementation. A well-written, easy to understand implementation, in addition to documentation and testing, is required in order for the code to be improved and refactored in subsequent work.
