   % daga motivations and goals, anonymous authentication, privacy (summary),
   Authentication, like every interaction involving parties with different roles and interests,
   can be seen trough the eyes and concerns of the different protagonists.
   In many situations, the verifying party (e.g.\ service provider) wants to hold their users accountable for their
   actions\sidenote[][-8pt]{
       e.g.\ in order to take actions to prevent dishonest behavior against their policies (such as vandalism, unfair usage, multiple accounts etc.) or the law.
   } and needs to be sure that access to the service is only granted to authorized people\sidenote{
       e.g.\ an online newspaper might want to be sure that the user is a subscriber or has purchased a plan.
   }.
   % TODO proportionality and corroboration example: it also sometimes need .. (e.g. whistleblowers, sybil etc..)
   While being legitimate those concerns can be seen as somewhat incompatible (or competing) with the growing desire (or necessity) for \textbf{privacy}
   of the proving party (user).\sidenote{

      e.g.\ users might agree with the needs of the provider without wanting to blindly trust the said provider
      won't abuse the 'functionality' and take it as an excuse to track, profile and sell their identified habits.
      or another class of users (whistleblowers or voting citizens or just anyone \ldots) might disagree on the disclosure of any other information
      besides "I am legitimate" from the authentication process.
      % TODO in fact not a really good example since daga linkage tag can enable a provider to profile a user ("anonymously")
      % and later nothing can prevent the link to an offline identity using correlation or other "side" ways ..with some confidence..
      % => need short epoch/rounds but then .. is it still useful to the verifying side ?
      % TODO can cite lindell on privacy etc.. and stress that DAGA offers some functionnalities that can be tuned or not etc..(context usage and lifetime etc.)
      % ==> better in background the linkability issue too..
      % ==> maybe introduce some privacy issues related to identity management then details DAGA properties and insert word regarding link issue
   }
   Hence sometimes, depending on the scenario, we would like to have authentication schemes that are more privacy aware
   than (e.g.\ ) the traditional identifier-challenge systems, i.e.\ schemes that are built with more flexible tools and
   that offer a better or fairer trade-off between the competing needs of both sides.
   \textbf{D}eniable \textbf{A}nonymous \textbf{G}roup \textbf{A}uthentication (DAGA)
   is one of such anonymous authentication protocols that was designed by Ewa Syta during her Ph.D.\ thesis\sidecite{syta_identity_2015}
   and was first described in \sidecite{syta_deniable_2014}.
   It "allows a user to authenticate as an anonymous member of \ldots\ a group defined by a list of public keys"\cite{syta_identity_2015},
   is decentralized and offers a set of security features that will be detailed later (see~\ref{sec:daga}).


   % FIXME (s): goals history ? explain not clear from start (especially POP ?) etc..
   % give explanations why "research/accumulator" part not done ? (time, not reasonable etc..discussion with Ewa..) previous state context/labo,

   The present work consists in first integrating DAGA into the \emph{cothority}\sidenote{see~\ref{subsec:cothority}} framework
   by creating a new DAGA authentication service that aims to be easily (re)usable,
   i.e.\ not tailored to a specific scenario and offering only vanilla DAGA authentication and context creation protocols.
   %(TODO is it clear that challenge protocol/proof included in "authentication" ?)
   Then using the new DAGA cothority to build a proof of concept login service consisting in an
   OpenID Connect (OIDC) Identity Provider offering DAGA authentication as a service.
   This allows every OIDC aware clients (or said differently: everyone) to delegates their user authentication to the DAGA cothority
   by mean of a well proven and established standard.
   % TODO sentence describing "report roadmap"  + cross ref
