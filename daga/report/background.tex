    % mostly DONE exept ev. menpage, 1 figure, ev more example (whistleblower), + link / cite other authentication systems

    In this chapter we will introduce various notions of interest such as authentication and identification
    before offering an introduction to DAGA.


    \section{Authentication, Identification and Privacy}
    \label{sec:authentication}

    % TODO add subsubsections if make more readable ?
    Informally authentication is the act or process allowing a verifying party to ensure that a claim coming from the other party
    is true i.e.\ the other party is indeed who or what it claims to be, while identification results in disclosing or verifying
    an (often offline) identity.
    The following section aims at clarifying those notions.
    Readers interested in more complete and extensive definitions and discussions on
    these complex and related issues can see\cite[p.~13-28]{syta_identity_2015} and\sidecite{lindell_anonymous_2011}
    who are addressing the subject in great depth and where used to offer the following overview.
    \subsubsection{\~Problem}
    Traditionally, authentication mechanisms whose purpose is to verify membership in a group
    (e.g.\ subscribers, users, moderators, citizens etc.)
    are often strongly linked with identification mechanisms, at least it is the case for the average John Doe experience
    who, e.g., was usually asked various information at enrollment time before being asked for its username or email
    (identification purpose) and password or challenge (authentication purpose) every time he wants to access the service.
    \sidenote{
        in such systems, the membership is decided "by (first) identifying an individual,
        then verifying that the individual is a member"\cite{goos_anonymous_1999} of the group.
    }
    \emph{Authentication} mechanisms are necessary when we want to implement \emph{Authorization} mechanisms, indeed we
    need to verify the trustworthiness of someone or something's answer to the question "why should we consider your request ?"
    before actually starting to wonder if we want to grant access to a resource or service.
    On the other hand, there are lots of use case where \emph{identification} is not needed at all, might be undesirable or even dangerous.
    For instance some services might require identification at enrollment time but not at authentication time\sidenote{
        e.g.\ in voting systems, you usually have to prove your identity in order to request a ballot but the vote itself is
        anonymous.
    } or sometimes identification is only there as a poor attempt \blockquote{
        to solve problems orthogonal to authentication and authorization, such as spam or misbehavior
    }\cite{syta_identity_2015}, or for not any particularly good reason at all (legacy, "simplicity", laziness).
    That is why, as already mentioned in the introduction, we need tools that allow service providers to build their identity managements
    systems in ways that take more into account the privacy of their end user and are fine-tuned to the specific needs of the application,
    systems where the trade-off between the competing interests of both side (user privacy vs provider confidence) is more balanced and sound.
    Because privacy has always been important\sidenote{
        Privacy preserves our "right to be let alone"\cite{warren_right_1890} and personal autonomy, while protecting us from targeted manipulation
        and self-censorship/inhibition.
    } and is probably more now we live in the era of big data and observe privacy breaches everywhere.
    % TODO more/better examples (whistleblowing) ?


    \subsection{Definitions}
    \label{subsec:definitions}

    To lift the veil on the seemingly paradoxical notions of anonymity and privacy when discussing authentication, we need to introduce
    more formal definitions of the terms and processes being discussed.\newline
    Authentication (and obviously identification) is closely related to the management of identities.
    An identity need first to be established at enrollment time before being verified at authentication time and consists
    informally of an \blockquote{ever-evolving\sidenote{%[][-2.5cm]{
        Indeed every authenticated action can enrich the initial set of information (from enrollment time) a provider knows about a client.
    } set of information about the client [/entity\sidenote{%[][-1.5cm]{
        An \textbf{entity} represents a user, object or resource and can have multiple identities
    }] the identity belongs to}\cite{syta_identity_2015}.

    % TODO maybe first say we can have 2 types of identites: identities and group identities
    More formally an \emph{identity} is defined to be a set of attributes\sidenote{%[][-1cm]{
        %or 'group identity'
        Attributes are properties of an entity, and a set of attributes, called a \emph{group identity}, specifies
        a \emph{group} \blockquote{consisting of exactly those entities satisfying all attributes in the group identity}\cite{syta_identity_2015}.
        % set of attributes = group identity = defines a possibly empty group
    } that specify a \emph{group}\sidenote{
        set of entities in an \emph{universe}, where an universe is a set of all entities of interests for a specific application)
    } containing a single entity.\cite{syta_identity_2015}
    We can see that those definitions allow us to make the distinction between attributes sets that describe a single entity (identity) and
    attributes sets that describes multiple entities at once (group identity), and this key distinction allows us to offer
    a more formal definition of \emph{identification} and \emph{authentication}.
    Thus Authentication becomes the process allowing a verifier to confirm that an entity (prover) is a member of a group,
    that is the process verifying that the entity possesses all the attributes of the group identity in question.
    Whereas identification become a special case of authentication where the group is a singleton.
    Hence the authentication process only needs to verify that entities possess attributes of interest to the verifying side
    and this set of attributes may describe a sufficiently large group to yields a satisfying amount of privacy to the proving side.
    \sidenote{concept somewhat related to k-anonymity\cite{sweeney_k-anonymity:_2002}}

    % TODOs small conclusion ?, link with next section DAGA, clarify problem and solutions
    % what is the point being made ? continue to clarify propos (already said somewhere 2times I'd say..)
    % definition offers flexibility, e.g can estabslish an identity at enrollement time but verify only a subset of attributes /group identity at authentication time
    % If we go back to the lifetime of identities and to service provider expectations/goals can fine tune etc..
    % examples: etc..offer new great possibilities, example of anon chat where all participant are verified at enrollement time (not bots, not sexual harrasser etc..cite lindell)

    \section{Deniable Anonymous Group Authentication (DAGA)}
    \label{sec:daga}

    Following the previously introduced definitions, this section summarize chapter 4 of~\cite{syta_identity_2015} to offer
    a tour of DAGA. For a more complete and formal description please refer to the original material.
    \newline
    DAGA is an \emph{authentication} protocol run between an \emph{entity} (user)
    at the proving side and a set of \emph{anytrust}\sidenote{
        the servers are "collectively but not individually trusted"\cite{syta_identity_2015}, that is we assume there is at least
        one honest server.
    } servers at the verifying side.
    The protocol allows the servers to verify that an entity is indeed a member of a \emph{group} specified by a list
    of public keys\sidenote{
        accordingly, an entity belongs to the group if it possesses/knows one of the corresponding private keys.
    }, while preserving the entity's anonymity in a forward-secure way.
    Moreover, authentication attempts are divided into epochs (or authentication rounds) and each successful authentication
    results in a unique (per-round) tag that allows linking authentications of the same entity thus offering to the verifying side
    the proportionality\sidenote{see below~\ref{def:proportionality}} property it sometimes expects/needs.
    \subsection[Properties]{Properties\sidenote{% FIXME I suspect this is the cause of the bad numbering of citations
        see\cite[Sect.~4.6]{syta_identity_2015} for complete and formal descriptions, assumptions and proofs.
    }} % -what are the properties of DAGA and why useful/wanted !! intro vs description vs etc..! pay attention to not tourne en rond
    \label{subsec:properties}

    DAGA offers the following properties. % TODO under those assupmtions: list ?
    \begin{description}%{forward anonymity}
        \item[Completeness] \hfill \\
            Entities that are member of the group and following correctly the protocol authenticate successfully\sidenote{
                Unless the process is aborted upon exposure of misbehaving or dishonest server(s).
                If nothing is done administratively to change the situation then the dishonest server(s) can prevent any
                progress.
            }
        \item[Soundness] \hfill \\
            Only entities that are member of the group can authenticate.
        \item[Anonymity] \hfill \\
            % ddh under random oracle model =>
            No PPT\sidenote{
                Probabilistic Polynomial Time, probabilistic (non deterministic) adversaries/algorithms/turing machines
                that are polynomially bounded (in time).
            } adversary can guess which member authenticated with probability bigger than random guessing.
        \item[Forward Anonymity] \hfill \\
            Once an authentication round (epoch) is ended, it is impossible to break the anonymity of any entity,
            even if the private keys of nearly all actors are compromised\sidenote{
                In its basic (and implemented) form, DAGA requires that the key of the honest server remains undisclosed
                (but this requirement can be relaxed).
            }
        \item[Proportionality] \hfill \\\label{def:proportionality}
            Authentication of an entity results in same final linkage tag for each authentications made during same epoch.
            (but authentications of same entity made during different epochs are unlinkable).
            The linkage tag act as anonymous or pseudonymous IDs of the entity during a round. % TODO corroboration whistleblowers example ? or as already mentioned..
        \item[Deniability] \hfill \\
            The entity can successfully deny having ever attempted to authenticate itself, in fact it is impossible to
            prove that any entity authenticated at all.\sidenote{
                The entity proves her membership (authenticates) by using an interactive zero-knowledge proof of knowledge, meaning that
                the interaction transcript cannot convince anyone but the original actors and is indistinguishable from
                a transcript obtained by running a simulator.
                %FIXME, point to where I say that the "cannot prove any authenticated" doesn't hold IMO..
                %(on success auth all servers obtains all the server's other proofs of correctness, those proofs are generated
                %only if server accepted req of client, at least one server honest => someone authenticated for sure
            }
        \item[Forward Deniability] \hfill \\
            The deniability property is retained even if an attacker gains additional knowledge of the private keys of all entities/users.
    \end{description}

    %\subsection{Existing Approaches}
    % TODO short comparisoin with LRS or other schemes ? => not time + maybe not my work => maybe point to syta paper.
    % The addition of forward-security and deniability properties (but remember of what I saw...about deniability) ... thus ..can be envisioned
    % maybe a chart of auth. methods/crypto primitives vs properties
    % other well known schemes such as hartemink99/lindell (not decentralized not deniable) etc..
    % or at least ! just link to relevant parts in syta

    % other attempts at resolving same issues with different properties : fdgfdfm, ONTAP, others

    \subsection[Overall description]{Overall description\sidenote{
        see\cite[Sect.~4.3]{syta_identity_2015} for complete description.
    }}

    % big picture description
    \subsubsection{Authentication Context}
    \label{subsubsec:dagacontext}
    As already mentioned, DAGA divides authentication attempts into epochs or authentication rounds.
    A client wanting to authenticate itself as a member of a group during an epoch needs such a group description
    as well as other round specific information which are packed in a public structure called an \emph{authentication context}.
    %\noindent\begin{minipage}{\linewidth} % prevent the thing to be breaked accross multiple pages
    \newline
    An authentication context \(C\) contains:
    \begin{description}[font=\boldmath]
        \item[\(\vec{X}\):]
            the long-term public keys of the \(n\) entities/users\sidenote{
                The keys used are Diffie-Hellman-Merkle (DH) key pairs.
            }
        \item[\(\vec{Y}\):]
            the long-term public keys of the \(m\) servers
        \item[\textbf{\(\vec{R} = \{R_1, \ldots, R_m\}\)}:]
            each server's commitment \(R_i=g^{r_i}\) to its per-round secret \(r_i\)
        \item[\(\vec{H}= \{h_1, \ldots, h_n\}\):]
            the unique per-round generators of \(\mathcal{G}\) associated to each clients\sidenote {
                "such that no one knows the logarithmic relationship between any \(h_i\) and \(g\) or between
                \(h_i\) and \(h_i'\) for any pair of clients \(i \neq i'\)"\cite{syta_identity_2015}
            }
        \item[\(\mathcal{G}\) and \(g\):]
            implicitly and obviously all actors know the algebraic group \(\mathcal{G}\)\sidenote{
                see~\autoref{subsec:library}
            } in usage and one common generator \(g\). (the same used in the above descriptions in multiplicative notation)
    \end{description}
    %\end{minipage}
    %\footnotetext{ % tricks to allow a sidenote inside the minipage
    %    "such that no one knows the logarithmic relationship between any \(H_i\) and \(g\) or between
    %    \(H_i\) and \(H_i'\) for any pair of clients \(i \neq i'\)
    %}

    % here description of protocol and proofs
    \subsubsection{Authentication}
    \label{subsubsec:dagaauth}
    Then, once in possession of the context, the \(i\)\textsuperscript{th} entity can start its authentication process
    which consists in building an initial linkage tag \(T_0=h_i^s\) using its per-round generator\sidenote{
        \(s=\prod_{j=1}^m s_j\), where \(sj=H(Y_j^{z_i})\) is a shared secret (with server j) derived from
        DH key exchange using a new ephemeral DH key \(Z_i=g^{z_i}\) and a suitable hash function \(H\).
    } as well as proving its membership and the correctness of its computation to the verifying servers.

    The servers are convinced by an interactive \emph{HVZK}\sidenote{
        Honest-Verifier Zero-Knowledge, see later~\ref{subsubsec:sigmaprotocol}
    } proof of knowledge\sidenote{
        designed and executed following techniques described by Camenish and Stadler in\cite{camenisch_proof_1997}
    } for the following "OR"-predicate~:\newline\centerline{
%        \(PK\{(x_i, s) : \vee_{k=1}^n (X_k=g^{x_k} \wedge S_m=g^s \wedge T_0=h_k^s)\}\)
        \(PK\{(x_i, s) : \vee_{k=1}^n ("knows\ k^{th}\'s\ secret\ key" \wedge "T_0\ correct")\}\)
    }
    \noindent
%    which states that he knows one private key and that it followed the protocol to build \(T_0\)\sidenote{
%        To see what \(S_m\) corresponds to, see the complete description in\cite{syta_identity_2015}.
%    }.
    Once done, the entity embeds the resulting tag and proof transcript in its authentication message \(M_0\) and send it
    to an arbitrary server.
    Then all the servers will, in turn (ring order based on indices in context), process the request, by each~:
    \begin{enumerate}[leftmargin=!,itemsep=-1ex]
        \item verifying the client's proof and all the previous (if not first) server's proof (aborting the process if either of them is invalid)
        \item scrubbing their shared secret \(s_j\) from the tag being built and replacing it by their per-round secret \(r_j\)
                to build an intermediate tag \(T_j=(T_{j-1})^{(r_j)(s_j^{-1})}\)
        \item adding a proof stating that they did their work correctly or exposing a misbehaving client (in that case they set \(T_j=0\))
    \end{enumerate}
    At the end \blockquote{
        all servers learn a final linkage tag \(T_f=h_i^{\prod_{k=1}^m r_k}\)
    }\cite{syta_identity_2015} in case of success or \(T_f=0\) in case of failure.
    Obviously communication between all actors need to take place over secure authenticated channels
    and if we don't want to destroy the purpose of DAGA we need an anonymous routing mechanism\cite{lindell_anonymous_2011}.

    %- FIXME syta figure or not..or previous student here or later in implementation architecture !! cross ref ?
    %anonymity is protected as long as there are respectively one and two\sidenote{
    %"anonymity is trivially impossible if n-1 [(all other)] clients choose to collude
    %against only one honest client."\cite{syta_identity_2015}
    %} honest server and entity/client and that their private keys are not compromised.