   
    This part builds upon the work of a previous student available on Github.\sidecite{villard_deniable_2017}
    The library is implemented in golang\sidecite{pike_go_nodate} and uses heavily the Kyber
    cryptography library developed at EPFL by the DEDIS team\sidecite{dedis_advanced_nodate}.\newline

    The goal of this part of the project was to ultimately integrate the new DAGA library into Kyber.
    The previous work was first ported from kyber.v0 (Crypto.v0) to kyber.v2, before being reworked to address some issues.\medskip

    Since the goal was to integrate the library into Kyber and that one of the main point of Kyber is to \blockquote{
    facilitate upgrading applications to new cryptographic algorithms or switching to alternative algorithms
    for experimentation purposes}\footnotemark[\thefootnote]
    we decided to try to leverage as much as possible of the facilities offered by Kyber.

    The first step in this direction was the addition of a new Suite\sidenote{
        A Suite is \blockquote{
            an abstract interface to a full suite of (\ldots) crypto primitives chosen to be suited to
            each other and have matching security parameters.
        } quoted from \href{https://github.com/dedis/kyber/blob/v0/abstract/suite.go}
    } interface for DAGA, that will notably allow future users to switch to other cryptographic building blocks if needed or wanted.
    For instance the original paper  worked in a Schnorr group\sidenote{
        specifically, the group of quadratic residues modulo a safe prime.
    } (i.e.\ a large prime-order subgroup of the multiplicative group of integer modulo \(p\) where \(p\) is prime),
    while the previous and current work implementation works with a prime-order group constructed with the twisted Edwards curve birationally equivalent to
    Curve25519\cite{yung_curve25519:_2006}(i.e.\ the suite uses the same group that is used in Ed25519's variant of the EdDSA signature scheme).

    We can see that from DAGA's perspective either implementation is fine as long as it satisfy DAGA's assumptions on the
    algebraic group being used, which is the case.\sidenote{
        DAGA is heavily based on discrete logarithm cryptography,
        thus DAGA assumes a group \(\mathcal{G}\) where the decision diffie-hellman problem\cite{boneh_decision_1998}
        is assumed to be hard (DDH assumption).
    } But from an engineering perspective this means we need to identify the individual high-level operations that are needed by DAGA
    and that would need to be dealt with different level of care depending on the underlying group and primitives used and their implementation in Kyber.
    (e.g.\ for diffie-hellman key exchange, we usually need to validate the public keys.\sidecite{valenta_measuring_2017}
    but with current EC concrete implementation we, at least in theory, don't need to care)\sidenote{
        In the current curve if the secret keys are correctly generated to be multiple of 8 (the curve's cofactor) then
        we don't have to check if the remote public key is in the correct subgroup, because even if it is not,
        the DH shared secret cannot leak (by construction) any bits of the targeted secret key.
        However in practise it is not the case, we noticed that the code generating the secret keys in Kyber is, at the time of writing, not correct,
        still seems that it is not a big deal because I am under impression that bad Points are rejected / not representable
        by the library. % when I think of the amount of time I dedicated to these crypto issues and that I'm just writing shit by lack of time and energy...:(
    }\medskip % TODO if not used to make link with lowlevel crypto concerns maybe dont mention it at all

    As a second major step we can mention, the rewriting of the client-side code and notably the PKClient proof\sidenote{
        see section~\ref{subsubsec:dagaauth}
    } related code to make use of the existing kyber.Proof framework instead of rewriting whole proof machinery by hand.

    This was done for multiple reasons, readability and maintainability for a start, it lessen the bug surface, i.e.\ we only need
    to look for proof related code at one well written, tested and documented place (the kyber.Proof package).
    Then we can view the same rationale from another perspective, since the proof framework is part of Kyber and we eventually
    want to integrate DAGA into Kyber, we might give a try at the "eat your own dog food" motto and use it in our own projects.
    To expand on this topic, we had to design some API wrappers around the proof framework to interface with it in a more
    easy and familiar way.\sidenote{
        see the clientProverCtx, clientVerifierCtx and their methods in client\_proof.go.
    }
    Those wrappers could maybe be adapted into an additional reusable API/pattern for the proof framework to ease future
    usage where, like in our case, the user needs to build interactive proofs across different machines.\sidenote{
        they allow to synchronize easily with a running proof.Prover, to extract the commitments in order to forward them elsewhere,
        then to continue the proof once a challenge was received from the remote end, etc.
    }

    Finally we introduced more documentation with links to original DAGA paper, better error messages and minor refactoring to
    make the code more readable DRY and idiomatic.\newline
    Still in our opinion the server side code is not production ready, nor of same quality level as the remaining of Kyber
    and thus should be mostly rewritten too (including the server proofs).
    This has not been done because the implementation was good enough for our current goal and we moved
    on the next phases in order to not lose more time on it.
    The library is well tested but the tests are mostly unmaintainable and it would be easy to miss some important cases.
    They were refactored to introduce the proper usage of testify and to fix some easy things but they were ultimately left in
    the state they were found.
    This leaves the library in an unfinished state and for all of these reasons and the fact that I don't fully endorse it
    in its current state the integration in Kyber was not performed.\sidenote{
        at some point after the port to kyber.v2 a WIP pull request was made and ultimately closed for the same reasons.
    }
    Still the code is full of comments detailing the direction to take to resolve the situation.


    % TODOs maybe if time and relevant => probably not
    %    -list some of design choices, such as interfaces (client server etc..)to allow multiple concrete implementation in user code (that need other things than pure daga)
    %      list example of benefit and usage ? favorized instead of struct to allow eaysier testing etc (functions accept interface) follow best practises


    % conclude with the fact that I thought about lots of corner cases and implementation issues and identified things that I'd like to see be done but not time
    % -add validations etc.. try to implement recommended practises regarding DH to try to avoid confinements and small subgroup attacks => nope i don't have the energy...


    %-IMO not sure daga should be put in kyber (and for sure not in sign/daga...) its a thing built with kyber but..


    %-since currently there there are no schnorr group in kyber and such schnorr group implementation is likely to be more costly
    %maybe the whole suite idea is finally a bad idea and maybe integrating daga in kyber is a bad idea too


    % had to go for the best given time constraints etc..
    % at some point was told that it is not my job (not very clear from start and divergent opinion of my two supervisors)
    % and more importantly that I won't have time to do the rest if I continue rewriting daga (which we all aggree i'd say)
    % (already huge amount of work, sic syta etc..even if not totally acknowledged by linus that is disapointed I didnt do more (how ???)
    % that's why the work is kinda left in an unfinished state with lots of comments and todos

    % pfff :(
