\section{Known bugs}
In this section I will describe the bugs that are either present in the application or have been fixed but are worth to be known. As some problems are related to JavaScript modules or directly to NativeScript, it's possible that the fix may not be necessary anymore in the future. This also allows understanding the necessity of the Makefile: by now, every solved bugs that concerns inherent issues of using NativeScript are fixed using patches, which are applied during the installation with the Makefile. For future bug fixes, this approach is recommended as it keeps the structure of the appli\-cation consistent.

\subsubsection*{Already known bugs}
The list of bugs stated in the report describing the first implementation of CP-MAC are still relevant. More precisely, the WebSocket problem on iOS has still not been fixed yet. The complete list can be found in \cite{petrimaire2018}.

\subsubsection*{KyberJS dependencies}
Sometimes, NPM packages such as KyberJS use built-in Node modules (\texttt{crypto}, in our case) that aren't just JavaScript code but make use of primitives that are specific to the platform on which they run (the browser, or Node). For some of them, an equivalent can be found for NativeScript, such as \texttt{nativescript-crypto} for the previous example. But the faulty modules that imports platform-specific packages still need to update all their \lstinline[language=JavaScript]|require()| statements to point to the correct versions. This can be handled by the module \texttt{nativescript-nodeify}, which installs a NativeScript hook\footnote{Hooks are scripts that are executed at specific moments, such as before the compilation begins.} that will traverse every modules and sub-modules to find incorrect \lstinline[language=JavaScript]|require()| calls to non-compatible packages and replace them with their NativeScript version.

However, in his current version\footnote{0.7.0 as of June 2018.}, \texttt{nativescript-nodeify} excludes every package containing an Organization scope, which prevents KyberJS from being converted, as KyberJS package is \texttt{@dedis/kyber-js}. An issue has been opened on GitHub\footnote{\url{https://github.com/EddyVerbruggen/nativescript-nodeify/issues/33}.} to find out why such a rule is applied, but no answer has been given yet. Thus, a patch applied by the Makefile has been created to remove this condition. Future versions of \texttt{nativescript-nodeify} will maybe fix this issue.

\subsubsection*{iOS version delay}
Unfortunately, due to a bug solved very lately during the project, the iOS version of CP-MAC is not as polished as the Android version. Particularly, some details on the UI still need some work, and some instabilities due to the lack of tests are present. However, some work on this platform is planned and will probably fix these details soon. 

The bug that caused this delay was due to the way \texttt{UglifyJS} mangled\footnote{Mangling is the process of optimizing JavaScrit code size, for example by renaming variables or functions to single letters.} the module CothorityJS. The version of the JavaScriptCore used on iOS was reporting two variables for having the same name, though actual JavaScript specifications on variable scope should allow this case. This was fixed by setting some specifics flags on \texttt{UglifyJS} to prevent this situation\footnote{More details can be read at \url{https://bugs.webkit.org/show_bug.cgi?id=171041}.}.

