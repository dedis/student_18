\section{Planned improvements}
The first project dedicated to CP-MAC focused on implementing the building blocks of the application and led to a functional library containing all the necessary parts to continue its development:  PoP-Party management (crea\-tion, publication, attendees registration and party finalization), user mana\-gement, and so on. However, as the major part of the work was dedicated to this huge library, the application main weakness resides in his usability. I will then present the main points that I had to work on to improve this weakness.

\begin{description}[style=nextline]
	\item[Interface]  As user interface couldn't be improved in the initial implementation of CP-MAC due to lack of time, it has been decided to define new guidelines and implement consistent rules about UI throughout the application. More details can be found at section \ref{sec:interface}: \nameref{sec:interface}.
	\item[Cothority v2] The work that has been done on Cothority since the first version of CP-MAC is quite consequent, and several changes to the API appeared. It was then important to make the application consistent. The application now supports Cothorithy v2 standards, such as TLS addresses, hexa\-decimal keys (instead of previous Base64 keys), and so on. Also, KyberJS and CothorityJS (a Node module that takes care of all the communications between CP-MAC and a conode) are now used in the application and replace the current specific CP-MAC implementations, allowing a more homogenized framework utilization. 
	\item[Proof-Of-Persoonhood enhancements] Until now, the application allowed the organization of only one party at a time and the support for the attendee part was very limited. One of the idea that popped up was the addition of multiple parties support (for attendees and organizers), with intuitive status tracking (in configuration, finalized, ...). 
	
	Attendees should also be able to generate their PoP-Tokens and make use of them (i.e sign messages). 
	
	Finally, the way the party configuration is shared between the organizers had to be switched from the current (transitional) way that uses the famous text storing service PasteBin\footnote{\url{https://pastebin.com}} to a more seamless solution. Details about these improvements are available at section \ref{sec:pop_part}: \nameref{sec:pop_part}.
	\item[Proof of concept] In order to give CP-MAC a more usable feeling, a simple proof of concept has been designed: it starts from a long-running joke at the DEDIS laboratory called BeerCoin, where a group of person could claim a free beer per day/week/month at the expense of the laboratory. The usage of Proof-Of-Personhood would then be perfectly fitted, as it would ensure that one member can only have one beer per time period, without revealing which member already had his beer. The implementation of this feature is detailed in section \ref{sec:beercoin}: \nameref{sec:beercoin}.
\end{description}